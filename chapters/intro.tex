%-*-coding: utf-8-*-

\startprefacepage
	Основной целью данной работы является создания инструмента для ускорения процесса поиска \enquote{узких} мест в программах.  

	В современном мире все чаще внимание уделяется скорости работы программ. Пользователям важен быстрый ответ. Программы должны уметь справляться с обработкой большого, постоянно меняющегося, входного объема данных. Им нужно уметь обрабатывать все данные и совершать нужные действия в зависимости от них, пока не пришли новые. Если приложение перестает справляться с нагрузкой, то некоторые запросы не будут обработаны. Из-за чего ухудшается качество написанного продукта.

	Программы, которые анализируют поток интернет трафика, например DPI \cite{dpi}, очень сложно поддаются анализу и поиску критических мест стандартными средствами. Так например долго работающая программа в течении дня, может в короткий промежуток времени начать \enquote{тормозить}. Из-за чего она не успевает отвечать на новые запросы и происходят потери данных, которые были посланы на сетевую карту. Стандартные средства для анализа приложений написаны таким образом, чтобы использоваться для программ запущенных на короткое время и зачастую нужно им на вход передавать специальные данные, которые были бы показательными и могли привести к нахождению проблемных участков кода.

	Стандартные средства не подходят для  по нескольким причинам. Некоторый из них сильно замедляют работу, потому их нельзя запускать на реальном приложении. Некоторые могут привести программу в нерабочее состояние. По отчетам многих из них сложно понять, где в программе возникли проблемы. А также они считают время занимаемое каждой функции в среднем, за весь промежуток работы. Это не позволяет смотреть на состояние системы в фиксированные моменты времени.
    
    В связи с чем возникает задача написания своего инструмента для анализа производительности. Он должен уметь собирать информацию с приложения, а также отображать ее пользователям. Инструмент должен быть прост в использовании. Должна быть возможность смотреть состояние программы на определенный промежуток времени.  
    
    Написанный инструмент поможет выявить проблемные участки кода. Исправление которых ускорит приложение и поможет ему обрабатывать весь поток входящих данных без потерь.

	Первая глава работы посвящена обзору существующих решений. Там же обсуждаются проблемы текущих решений и формулируется постановка задачи. 

	Во второй главе будет рассмотрена архитектура инструмента по сбору информации о приложении. Трудности возникающие при его написании.

	В третьей главе будет показан инструмент для визуализации собранных данных и рассказаны основные моменты возникающие при его реализации.

В заключении  показано текущее состояния разработки и рассмотрены перспективы ее развития. 
