#include <bits/stdc++.h>
#include <thread>
#include <sstream>

using namespace std;
using namespace chrono;

struct logger {

    logger(string const & log_path = "log") 
        : _stopped(false)
    {
        _logger.open(log_path);
    }

    void start() {
        _self = thread(bind(&logger::run, this));
    }

    void stop() {
        _mutex.lock();
        _stopped = true;
        _cv.notify_one();
        _mutex.unlock();

        _self.join();
    }

    template<typename T>
    logger& operator<<(T data) {
        stringstream ss;
        ss << data;

        string for_buff(ss.str());

        unique_lock<mutex> lk(_mutex);
        _buff.insert(_buff.end(), for_buff.begin(), for_buff.end());
        _cv.notify_one();

        return *this;
    }

private:
    void run() {
        while (!_stopped) {
            unique_lock<mutex> lk(_mutex);

            _cv.wait(lk, [this]() { return _stopped || !_buff.empty(); });

            _logger.write(_buff.data(), _buff.size());
            _buff.clear();
        }
    }

    mutex _mutex;
    condition_variable _cv;

    ofstream _logger;
    vector<char> _buff;

    bool _stopped;
    thread _self;
};

int main() {
    int const cnt_runs = 100;
    int const n = 1e5;

    auto start = high_resolution_clock::now();

    logger log;
    log.start();

    for (size_t run_num = 0; run_num < cnt_runs; ++run_num) {
        for (size_t i = 0; i < n; ++i) {
            log << "hello world: " << i << "\n";
        }
    }

    log.stop();

    auto end = high_resolution_clock::now();

    cerr << "time in seconds: " << duration_cast<nanoseconds>(end - start).count() * 1.0 / cnt_runs / 1e9 << endl;

    return 0;    
}
